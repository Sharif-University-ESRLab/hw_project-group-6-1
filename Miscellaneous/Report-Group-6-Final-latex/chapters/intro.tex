\chapter{مقدمه}
محصول نهایی این پروژه، یک سیستم دید در شب است که درون اتومبیل قرار می‌گیرد و به راننده در هنگام رانندگی در تاریکی، کمک به سزایی می‌کند. در این سیستم اطلاعات از طریق یک دوربین حرارتی به ماژول رزپری منتقل می‌شود و کدهایی که در رزپری قرار داده شده است با انجا پردازش تصویری ساده، تشخیص خواهد داد که آیا موجود زنده‌ای در میدان دید راننده حضور دارد یا خیر. همچنین از طریق چراغ و صدا نتیجه را به راننده اطلاع می‌دهد.
\\

به صورت دقیق‌تر، این محصول با کمک یک دوربین مادون قرمز، می‌تواند دمای موانع در سر راه راننده را از فاصله‌ی دور تشخیص دهد. سپس اگر دمای قسمتی از تصویر روبه‌رویش نسبت به دمای محیط به مقدار نسبتا قابل ملاحظه‌ای بالاتر باشد، سیستم متوجه حضور یک موجود زنده شده و شروع به هشدار دادن به راننده می‌کند. نکته‌ای که وجود دارد این است که اگر این تغییر دما آن قدر بالا باشد که دیگر نتواند به عنوان دمای واقعی بدن یک موجود زنده در نظر گرفت، آنگاه سیستم نیز موجود زنده‌ای را شناسایی نمی‌کند زیرا تنها محدوده‌ی دمایی خاصی‌ست که می‌توان مربوط به دمای بدن موجودات زنده باشد.

مزیت رقابتی اصلی این محصول هزینه‌ی پایین ساخت آن است که با تغییر در بعضی از ماژول‌های سیستم، حتی می‌توان به هزینه‌ی کمتر نیز رسید. همچنین محصول نهایی بسیار کوچک خواهد بود زیرا به جای چراغ‌هایی که در نمونه‌ی اولیه‌ی ما استفاده شده است، در واقع باید چرا‌غ‌های خود اتومبیل قرار بگیرد و چون روشن کردن این چراغ‌ها از طریق ارتباط با کامپیوتر ماشین ممکن است در نتیجه کافی‌ست پس از انجام پردازش‌ها توسط رزپری و سنسور‌ها، دستور روشن شدن چراغ‌ها را به کامپیوتر اتومبیل ارسال و آن‌ها را روشن کرد.