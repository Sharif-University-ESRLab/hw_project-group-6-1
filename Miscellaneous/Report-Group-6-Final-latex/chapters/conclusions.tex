\chapter{جمع‌بندی}
در این پروژه به پیاده‌سازی سیستم دید در شب اتومبیل پرداختیم که به راننده برای جلوگیری از سانحه در محیط‌های تاریک کمک می‌کند. در این سیستم با استفاده از سنسور دوربین حرارتی که از طریق جذب مادون قرمز عمل می‌کند و همچنین واحد پردازشی رزپری‌پای، حضور یک موجود زنده جلوی دید راننده را تشخیص و به کمک چرا‌غ‌ها به اون هشدار دادیم. 

در کنار طراحی کلی و نحوه پیاده‌سازی، تمام کد‌های مربوط به این محصول نیز به صورت متن باز در گیتهاب پروژه قراره گرفته است و هرکسی می‌تواند با کمک این کدها به بهبود و ارتقاء این سیستم کمک کند و یا با الهام از آن، پیاده سازی خاص خودش را ارائه دهد. 

آنچه محصول ما را از دیگر محصول‌ها متمایز می‌کند قیمت بسیار پایین تمام‌شده‌ی ‌آن است که می‌توان حتی بیشتر آن را کاهش داد. به عنوان مثال با استفاده از آردوینو به جای رزپری و همچنین استفاده از چراغ‌های خودرو به جای چراغ‌های مجزا برای سیستم. حتی می‌توان در هنگام پیاده‌سازی این سیستم برای خودرو، کد آن را به عنوان یک نرم افزار در اختیار کامپیوتر خودرو قرار داد و تمام مسئولیت‌های رزپری را به کامپیوتر اتومبیل سپرد.